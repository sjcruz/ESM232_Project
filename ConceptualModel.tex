\documentclass[]{article}
\usepackage{lmodern}
\usepackage{amssymb,amsmath}
\usepackage{ifxetex,ifluatex}
\usepackage{fixltx2e} % provides \textsubscript
\ifnum 0\ifxetex 1\fi\ifluatex 1\fi=0 % if pdftex
  \usepackage[T1]{fontenc}
  \usepackage[utf8]{inputenc}
\else % if luatex or xelatex
  \ifxetex
    \usepackage{mathspec}
  \else
    \usepackage{fontspec}
  \fi
  \defaultfontfeatures{Ligatures=TeX,Scale=MatchLowercase}
\fi
% use upquote if available, for straight quotes in verbatim environments
\IfFileExists{upquote.sty}{\usepackage{upquote}}{}
% use microtype if available
\IfFileExists{microtype.sty}{%
\usepackage{microtype}
\UseMicrotypeSet[protrusion]{basicmath} % disable protrusion for tt fonts
}{}
\usepackage[margin=1in]{geometry}
\usepackage{hyperref}
\hypersetup{unicode=true,
            pdftitle={Conceptual Model},
            pdfauthor={Seleni Cruz and Juliette Verstaen},
            pdfborder={0 0 0},
            breaklinks=true}
\urlstyle{same}  % don't use monospace font for urls
\usepackage{longtable,booktabs}
\usepackage{graphicx,grffile}
\makeatletter
\def\maxwidth{\ifdim\Gin@nat@width>\linewidth\linewidth\else\Gin@nat@width\fi}
\def\maxheight{\ifdim\Gin@nat@height>\textheight\textheight\else\Gin@nat@height\fi}
\makeatother
% Scale images if necessary, so that they will not overflow the page
% margins by default, and it is still possible to overwrite the defaults
% using explicit options in \includegraphics[width, height, ...]{}
\setkeys{Gin}{width=\maxwidth,height=\maxheight,keepaspectratio}
\IfFileExists{parskip.sty}{%
\usepackage{parskip}
}{% else
\setlength{\parindent}{0pt}
\setlength{\parskip}{6pt plus 2pt minus 1pt}
}
\setlength{\emergencystretch}{3em}  % prevent overfull lines
\providecommand{\tightlist}{%
  \setlength{\itemsep}{0pt}\setlength{\parskip}{0pt}}
\setcounter{secnumdepth}{0}
% Redefines (sub)paragraphs to behave more like sections
\ifx\paragraph\undefined\else
\let\oldparagraph\paragraph
\renewcommand{\paragraph}[1]{\oldparagraph{#1}\mbox{}}
\fi
\ifx\subparagraph\undefined\else
\let\oldsubparagraph\subparagraph
\renewcommand{\subparagraph}[1]{\oldsubparagraph{#1}\mbox{}}
\fi

%%% Use protect on footnotes to avoid problems with footnotes in titles
\let\rmarkdownfootnote\footnote%
\def\footnote{\protect\rmarkdownfootnote}

%%% Change title format to be more compact
\usepackage{titling}

% Create subtitle command for use in maketitle
\newcommand{\subtitle}[1]{
  \posttitle{
    \begin{center}\large#1\end{center}
    }
}

\setlength{\droptitle}{-2em}

  \title{Conceptual Model}
    \pretitle{\vspace{\droptitle}\centering\huge}
  \posttitle{\par}
    \author{Seleni Cruz and Juliette Verstaen}
    \preauthor{\centering\large\emph}
  \postauthor{\par}
      \predate{\centering\large\emph}
  \postdate{\par}
    \date{April 29, 2019}


\begin{document}
\maketitle

\subsection{Recovery times and syncronized spatial management of
Fisheries}\label{recovery-times-and-syncronized-spatial-management-of-fisheries}

\paragraph{Motivation}\label{motivation}

Marine reserves are used widely to meet a myriad of objectives including
biodiversity protection and fisheries management. One advantage of
marine reserves over traditional fishery management is that reserves
protect not only target species but also habitat and non-target species
(Micheli et al., 2004). However, marine reserve theory has largely
focused on single species recovery. Few studies have focused on the
effects of marine reserves on multi-species interactions and how these
interactions can affect population recovery time and maximize fishery
benefits (Baskett, Micheli \& Levin, 2007; Shamhouri et al., 2017). We
chose to focus on predator-prey interaction as ``fishing down the
foodweb'' is a common phenomnenon in fisheries. As abundances of
organisms in higher tropic levels are reduced (in many cases due to
fishing), effort is then focused on lower trophic levels gradually
moving down the food web (Pauly et al. 1998).

\emph{Our guiding question is:} How does taking predator-prey
interactions into account in marine reserve thoery influence their
recovery times and thus fishery benefits? How do these results compare
to other single species management measures (such as single species
closures)? We re-create models presented in Samhouri et al. 2017 as a
basis and introduce marine reserves as an additional management
intervention.

\emph{Our main goal:} Identifying how biomass of predators and prey will
react to managment interventions in the form of various sizes of marine
reserves. The output will include a table and graphs with prey abdunance
(\#individuals), predator abundance (\# individuals), size of reserve
(\% of total area), and time (year).

\emph{Target audience:} We hope this model will be beneficial to
resource mangaers and planners who use marine reserves with the main
goal of multiple fisheries recovery.

\paragraph{Sub models}\label{sub-models}

Submodels will include: 1. Predator-prey model: To simulate interaction
between predator and prey, including growth, harvest and competition
among both species. This model will be lumped, dynamic, stochastic and
abstract. Inputs will include biological parameters such as intrinsic
growth rate, carrying capcity and competition coefficients, as well as,
harvest rates. The model will utliize generalist predator-prey
variables, thus sensitivity analysis will be conducted for intrinsic
growth rates.

Interactions of generalist predator prey dynamics are mathematicaly
described as follows:

\[\frac{dX}{dt}=r_XX(1-\frac{X}{K_X})- a_XPX-h_XX\]
\[\frac{dP}{dt}=P[c(a_XX + a_XY)-d_P](1-\frac{P}{K_P})- h_PP\]

Here, X and P denote prey and predator abundances in number of
individuals, \(r_X\) is the prey's intrinsic per-capita growth rate
(units: yr−1), \(K_X\) is the prey's logistic growth carrying capacity
(units: number of individuals), \(d_P\) is the predator's per capita
mortality rate (units: yr−1) and \(K_P\) is the predator's carrying
capacity (units: number of individuals) reflecting limiting factors
other than prey availability, such as habitat. The predator feeds on
prey X and Y with linear type I functional responses at per-capita rates
\(a_X\) and \(a_Y\), respectively (units: number of individuals−1 ×
yr−1), the relative magnitude of which reflects its preference for the
two prey, and converts these to predator biomass at rate c (units: prey
per predator). The predator and focal prey are harvested at constant
per-capita rates, \(h_P\) and \(h_X\) (units: yr−1). Adopted from
Samhouri et al., 2017.

\begin{longtable}[]{@{}lll@{}}
\caption{Table 1. List of Parameters}\tabularnewline
\toprule
Varible & Description & Units\tabularnewline
\midrule
\endfirsthead
\toprule
Varible & Description & Units\tabularnewline
\midrule
\endhead
P & Predator & \# individuals\tabularnewline
X & Focal prey & \# individuals\tabularnewline
Y & Non-dynamic prey & \# individuals\tabularnewline
rx & Growth rate prey & y\^{}-1\tabularnewline
dp & Predator mortality rate & y\^{}-1\tabularnewline
kp & Predator carrying capacity & \# individuals\tabularnewline
ax & Predation rate on focal prey & \#
individuals\^{}-1*y\^{}-1\tabularnewline
ay & Predation rate on non-dynamic prey & \#
individuals\^{}-1*y\^{}-1\tabularnewline
hp & Predator harvest rate & y\^{}-1\tabularnewline
hx & Focal prey harvest rate & y\^{}-1\tabularnewline
\bottomrule
\end{longtable}

\begin{enumerate}
\def\labelenumi{\arabic{enumi}.}
\setcounter{enumi}{1}
\item
  MPA model - A patch model to simulate predator-prey interactions given
  spatial closures. Space will be represented by a series of vectors.
\item
  Economic Model
\item
  Wrapper of these three models
\item
  Identify the characteristics of submodes that you will use - spatial
  or lumped, dynamic or static, deterministic or stochastic, physically
  based/abstract?
\item
  Identify the goal of your model - what is the question that it will be
  used to answer; for which types of users
\item
  Design a figure to illustrate your conceptual model
\item
  Determine the inputs, outputs that will be used for each submodes
\end{enumerate}


\end{document}
